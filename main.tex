\documentclass{mynotes}
\usepackage{mymacro}
\begin{document}
\tableofcontents
\chapter{Linear Transformations}
\section{Linear Transformations}
\begin{definition}[linear transformation]
let $V$ and $W$ be vector spaces over the field \F. A linear transformation from $V$ into $W$ is a function $T$ from $V$ into $W$ such that $$T(c\al+\be) = cT\al+T\be,$$ for all $\al,\be\in V$ and $c\in\F.$
\end{definition}
\begin{example}
Let $A$ be a fixed \mbyn matrix with entries in the field \F. The function $T$ defined by $T(x) = Ax$ is a linear transformation from $\F^{n\times1}$ to $\F^{m\times1}$. The function $U$ defined by $U(\al) = \al A$ is a linear transformation from $F^{1\times m}$ to $F^{1\times n}$
\end{example}
\begin{example}
Let \R{} be the field of real numbers and let $V$ be the space of all functions from \R{} to \R{} which are continuous. Define $T$ by $(Tf)(x) = \int_0^x f(t)\,\mathrm{d}t$. Then $T$ is a linear transformation from $V$ to $V$.
\end{example}
\begin{remark}
It's important to notice that if $T$ is a linear transformation from $V$ into $W$, then $$T(0_{{}_V}) = 0_{{}_W}.$$
\end{remark}
\begin{remark}
Linear transformation is actually defined to preserve linear combinations. That is 
$$T(c_1\al_1+c_2\al_2+\ldots + c_n\al_n) = c_1T(\al_1)+c_2T(\al_2)+\ldots + c_nT(\al_n)$$
\end{remark}
\begin{theorem}\label{t1}
Let $V$ be a finite dimensional vector space over the field \F, let $\{\al_1,\al_2,\ldots,\al_n\}$ be any ordered basis for $V$, let $W$ be a vector space over the same field \F{} and let $\be_1,\be_2,\ldots,\be_n$ be any vectors in $W$. Then there is precisely one linear transformation $T$ from $V$ into $W$ such that $T\al_j = \be_j$, for all $j=1,\ldots, n$.
\end{theorem}
\begin{proof}
Given $\al\in V$, there is a unique $n$-tuple $(x_1,x_2,\ldots,x_n)$ such that $$\al = x_1\al_1+x_2\al_2+\ldots+x_n\al_n.$$We define $T\al = x_1\be_1+x_2\be_2\ldots +x_n\be_n.$ Then $T$ is a well-defined rule for associating with each vector $\al\in V$ a vector $T\al\in W$. It's clear that $T\al_j = \be_j$ for each $j$ and $T$ is a linear transformation. \\
If $U$ is a linear transformation from $V$ into $W$ with $U\al_j=\be_j, j=1,2,\ldots,n$, then for the vector $\al = \sum_{i=1}^{n} x_i\al_i$ we have 
$$
U\al = U\left(\sum_{i=1}^nx_i\al_i\right)
	= \sum_{i=1}^nx_i\left(U\al_i\right)
	= \sum_{i=1}^n x_i\be_i.
$$So $U$ is exactly the same rule $T$ which we defined. This shows that the linear transformation $T$ with $T\al_j = \be_j$ for each $j$ is unique.
\end{proof}
\begin{remark}
The proof of Theorem \ref{t1} show us the way to actually get the transformation $T$. 
\end{remark}
\begin{problem}
The vector $\al_1 = (1,2)$, $\al_2 = (3, 4)$ form a basis for $\R^2$. $\be_1 = (3,2,1)$, $\be_2 = (6,5,4)$ are two vectors in $\R^3$. We now want to find a linear transformation $T$ such that $T\al_j=\be_j$.\\
We see that $(1,0) = -2(3,2,1) +(6,5,4)$, thus $T(1,0) = -2(3,2,1)+(6,5,4) = (0,1,2)$. Similarly, we can find $T(0,1)$. Then we know all about this $T$.
\end{problem}
\begin{example}\label{exF}
Let $T$ be a linear transformation from $\F^m$ to $\F^n$. By Theorem \ref{t1} we know that $T$ is uniquely determined by the sequence of vectors $\be_1,\be_2,\ldots,\be_m\in\F^n$ where $$\be_i = T\epsilon_i, \qquad i = 1,2,\ldots, m,$$Namely, if we have $$\al = (x_1,x_2,\ldots,x_m),$$then $$T\al = x_1\be_1+\ldots+x_n\be_n.$$If $B$ is the \mbyn{} matrix which has row vectors $\be_1,\be_2,\ldots,\be_m$, this says that $$T\alpha = \alpha B.$$
\end{example}
\begin{remark}
From Example \ref{exF} we show that we can give an explicit and reasonably simple description of all linear transformations from $\F^m$ to $\F^n$.
\end{remark}
\begin{remark}If $T$ is a linear transformatino from $V$ into $W$, then the range of $T$ is a subspace of $W$.
\end{remark}
\begin{remark}The set $N$ consisting of the vectors $\al\in V$ such that $T\al = 0$ is a subspace of $V$.
\end{remark}
\begin{definition}[null space, rank, nullity]
Let $V$ and $W$ be vector spaces over the field \F{} and let $T$ be a linear transformation from $V$ to $W$. The null space of $T$ is the set of all vectors $\al\in V$ such that $T\al = 0.$
If $V$ is finite-dimensional, the rank of $T$ is the dimension of range of $T$ and the nullity of $T$ is the dimension of the null space of $T$.
\end{definition}
\begin{theorem}
Let $V$ and $W$ be vector spaces over the field \F{} and let $T$ be a linear transformation from $V$ into $W$. Suppose that $V$ is finite-dimensional, then $$\rank{T} + \nullity{T} = \dim V$$
\end{theorem}
\begin{proof}
Let $\{\al_1,\ldots,\al_k\}$ be a basis for the null space of $T$. We can find $\al_{k+1},\ldots,\al_{n}\in V$ such that $\{\al_1,\al_2,\ldots,\al_n\}$ is a basis for $V$. We shall show that $\{T\al_{k+1},\ldots,T\al_n\}$ is a basis for the range of $T$. The vector $T\al_1,\ldots,T\al_n$ certainly span the range of $T$. Since $T\al_1,\ldots,T\al_k$ are zero, $\{T\al_{k+1},\ldots,T\al_n\}$ spans the range of $T$. To see they are also independent, suppose we have scalars $c_i$ such that $$\sum_{i=k+1}^n c_i(T\al_i) = 0.$$Then we have $T(\sum_{i=k+1}^n c_i\al_i) = 0$ and accordingly $\al = \sum_{i = k+1}^n c_i\al_i$ is in the null space of $T$. We then must have $c_i = 0$ for $i = k+1,\ldots, n$. So $\{T\al_{k+1},\ldots,T\al_n\}$ is a basis for the range, we are done.
\end{proof}
\begin{theorem}
If $A$ is an \mbyn{} matrix with entries in the field \F, then $$\mbox{row rank}(A) = \mbox{column rank}(A).$$
\end{theorem}
\begin{proof}
Let $T$ be the linear transformation from $\F^{n\times1}$ to $\F^{m\times1}$ defined by $T(x)=Ax$. Say $S$ is the solution space of the system $Ax=0$, then $\nullity{T} = \dim{S}$. So we have $$\dim{S}+\rank{T} = n.$$ Notice that $\rank{T}$ is actually the dimension of the column space of matrix $A$. Say the RREF of matrix $A$ is $R$ and the row rank for $A$ and $R$ is $r$. Let $R_1,R_2,\ldots,R_n$ be the columns of matrix $R$, then there are $r$ of these columns, say $R_{p_1},\ldots,\R_{p_r}$, have a single $1$ as their only non-zero entry. Therefore, considering the space $S$, there are $n-r$ free variables, which means $\dim{S} = n-r$. So we have $\mbox{column rank}{A}=\mbox{row rank}{A}$.
\end{proof}
\section{The Algebra of Linear Transformation}
\begin{theorem}
Let $V$ and $W$ be vector spaces over the field \F. Let $T$ and $U$ be linear transformations from $V$ into $W$. The function $(T+U)$ defined by $$(T+U)(\al)=T\al+U\al$$is a linear transformation from $V$ into $W$. If $c$ is any element of \F, the function $(cT)$ defined by $(cT)(\al) = c(T\al)$ is a linear transformation from $V$ into $W$. The set of all linear transformation from $V$ into $W$, together with the addition and scalar multiplication defined above, is a vector space over the field \F.
\end{theorem}
\begin{proof}Omit.\end{proof}
\begin{remark}
We denote the space of linear transformations from $V$ into $W$ by $L(V,W)$.
\end{remark}
\begin{theorem}
Let $V$ be an $n$ dimensional vector space over the field \F, and let $W$ be an $m$ dimensional vector spaces over the field \F. Then the space $L(V,W)$ is finite-dimensional and has dimension $mn$.
\end{theorem}
\begin{proof}
Let $\mathcal{B} = \{\al_1,\al_2,\ldots,\al_n\}$ and $\mathcal{B}'= \{\be_1,\be_2,\ldots,\be_m\}$ be ordered bases for $V$ and $W$, respectively. For each pair of integer $(p,q)$ with $1\leq p\leq m$ and $1\leq q\leq n$, we define a linear transformation $E^{p,q}$ from $V$ into $W$ by $E^{p,q}(\al_i) = \delta_{iq}\be_p$.\\ Let $T$ be a linear transformation from $V$ to $W$. $(A_{1j},A_{2j},\ldots,A_{mj})$ is the coordinate vector of $T(\al_j)$ in the ordered basis $\mathcal{B}'$, i.e., $$T(\al_j) = \sum_{i=1}^mA_{ij}\be_i$$ Our claim is $$T=\sum_{p=1}^m\sum_{q=1}^nA_{pq}E^{p,q}.$$Actually, Let $U$ be the linear transformation defined by RHS of the equation, then for each $j$, 
\begin{align*}U(\al_j) &= \sum_{p=1}^m\sum_{q=1}^nA_{pq}E^{p,q}\al_j\\ &= \sum_{p=1}^m\sum_{q=1}^nA_{pq}\delta_{jq}\be_p\\ &=\sum_{p=1}^mA_{pj}\be_p = T(\al_j).\end{align*}So $T=U.$ This means $\{E^{p,q}\}$ spans $L(V,W)$. Furthermore, $\{E^{p,q}\}$ are independent because if $T=\sum\limits_{p}\sum\limits_{q}A_{pq}E^{p,q}$ is the zero transformation, then for each $j$, $T(\al_j)= \sum\limits_iA_{ij}\be_i=0$. So $A_{ij}=0$ for every $i,j$.
\end{proof}
\begin{theorem}\label{costheorem}
Let $V,W,Z$ be vector spaces over the field \F. Let $U$ be a  linear transoformation from $V$ into $W$, and $T$ be a linear transformation from $W$ to $Z$, then the funtion $(TU)$ defined by $(TU)(\al) = T(U{(\al)})$ is a linear transformation from $V$
 to $Z$.
 \end{theorem}
\begin{proof}Omit.\end{proof}
\begin{definition}[linear operator]
Let $V$ be a vector spaces over the field \F. A linear operator on $V$ is a linear transformation from $V$ into $V$.
\end{definition}
\begin{remark}
We notice that in Theorem \ref{costheorem}, if $V=W=Z$, then $(TU)$ is also a linear operator on $V$, i.e., there is a `multiplication' operation defined by composition on $L(V,V)$. In addition, $(UT)$ is also defined, but in general $(UT)-(TU)$ is not zero transformation.
\end{remark}
\begin{remark}
If $T$ is a linear operator on $V$, then we can define $T^n = TTT\ldots T$ without confusion. Proof is omitted. For convenience, we define $T^0=I\mbox{(identity transformation)}.$
\end{remark}
\begin{lemma}\label{proAl}
Let $V$ be a vector space over the field \F, $U,T_1,T_2$ be linear operators on $V$, c is any elements in the field \F.
\begin{itemize}
	\item[1)] $U=UI=IU$
	\item[2)] $U(T_1+T_2) = UT_1+UT_2;(T_1+T_2)U=T_1U+T_2U$
	\item[3)] $c(UT) = (cU)T=U(cT)$
\end{itemize}
\end{lemma}
\begin{remark}
Lemma \ref{proAl} and Theorem \ref{costheorem} tell us that $L(V,V)$, together with cosposion, is what known as a linear algebra with identity.
\end{remark}
\begin{example}
Let $A$ be an \mbyn matrix and $T$ be a linear transformation defined by $T(X) = Ax$. Let $B$ be an $p\times m$ matrix and $U$ be a linear transformation defined by $U(Y) = BY$. Then \begin{align*}(UT)(X) &= U(T(X))\\&=U(AX)\\&=B(AX)=BAX.\end{align*}
\end{example}
\begin{remark}
The effect of cosposition of $U,T$ is multiplication of matrices $B,A$.
\end{remark}
\begin{definition}[invertible]\label{invdef}
A linear transformation $T$ from $V$ into $W$ is invertible if there exist a function $U$ from $W$ into $V$ such that $(UT)$ is the identity transformation on $V$ and $(TU)$ is the identity transformaton on $W$. In this case, $U$ is unique and we denote $U$ by $T^{-1}$.
\end{definition}
\begin{remark}
In Definition \ref{invdef}, $T^{-1}$ exists if and only if
\begin{itemize}
	\item[1.] $T$ is one-one. ($T\al=T\be\implies \al=\be$)
	\item[2.] $T$ is onto. (The range of $T$ is $W$)
\end{itemize}
\end{remark}
\begin{theorem}
let $T$ be a linear transformation from $V$ into $W$. If $T$ is invertible, then the inverse $T^{-1}$ is a linear transformation from $W$ into $V$.
\end{theorem}
\begin{proof}
Omit.
\end{proof}
\begin{remark}
We see that $T^{-1}U^{-1}$ is the left and right inverse of $UT$, therefore the inverse of $(UT)$ is $T^{-1}U^{-1}$.
\end{remark}
\begin{definition}[non-singular]
We call a linear transformation $T$ non-singular if $T\gamma=0\implies\gamma=0,$i.e., the null space of $T$ is $0$.
\end{definition}
\begin{remark}
Evidently, $T$ is one-one if and only if $T$ is non-singular.
\end{remark}
\begin{remark}
 Non-singular linear transformations are those which preserve linear independence, as the following theorem claims.
\end{remark}
\begin{theorem}
Let $T$ be a linear transformation from $V$ into $W$. Then $T$ is non-singular if and only if $T$ carries each linearly independent subset of $V$ onto a linearly independent subset of $W$.
\end{theorem}
\begin{proof}
Suppose that $T$ is non-singular. Let $S$ be a linearly independent subset of $V$. If $\al_1,\ldots,\al_k$ are vectors in $S$, then the vector $T\al_1,\ldots,T\al_k$ are linearly independent. For if $$c_1(T\al_1)+\ldots+c_k(T\al_k)=0$$then $$T(c_1\al_1+\ldots+c_k\al_k) = 0$$therefore $$c_1\al_1+\ldots+c_k\al_k = 0.$$Since $\al_i$ are linearly independent, we have for each $j=1,2,\ldots,k$, $c_j = 0$.\\
Suppose that $T$ carries independent subsets onto independent subsets. Then $T$ must be non-singular. For if $T\al=0$, and $\al$ is not $0$. Then an independent set $S$ consisting of $\al$ will have its image a dependent set.
\end{proof}
\begin{theorem}
Let $V$ and $W$ be finite-dimensional vector spaces over the field \F{} such that $\dim{V}=\dim{W}$. If $T$ is a linear transformation from $V$ into $W$, the following are equivalent:
\begin{itemize}
	\item[(i)]	$T$ is invertible.
	\item[(ii)] $T$ is non-singular.
	\item[(iii)] $T$ is on-to.
\end{itemize}
\end{theorem}
\begin{proof}
Let $n=\dim{V}=\dim{W}.$ Since $$\rank{T}+\nullity{T} = n.$$So if $T$ is non-singular, then $\nullity{T} = 0$ and $\rank{T}= n$, i.e., $T$ is on-to.
If $T$ is on-to, then $\rank{T} = n$ and therefore $\nullity{T} = 0$ ($T$ is non-singular). \\Therefore $T$ is non-singular if and only if $T(V)=W.$ So, if either condition (ii) or (iii) holds, the other is satisfied as well and $T$ is invertible.
\end{proof}
\section{Isomorphism}
\begin{definition}[isomorphism]
If $V$ and $W$ are vector spaces over the field \F, any one-one linear transformation $T$ from $V$ onto $W$ is called an isomorphism of $V$ onto $W$. If there exists an isomorphism of $V$ onto $W$, we say that $V$ is isomorphic to $W$.
\end{definition}
\begin{remark}
If $V$ is isomorphic to $W$, then $W$ is isomorphic to $V$.
\end{remark}
\begin{theorem}
Every $n$-dimensional vector space over the field \F is isomorphic to the space $\F^n.$
\end{theorem}
\begin{proof}
Let $V$ be an $n$-dimensional vector space over the field \F and let $\mathcal{B}=\{\al_1,\ldots,\al_n\}$ be a basis for $V$. We define a function $T$ from $V$ to $\F^n$, as follows: If $\al$ is in $V$, let $T\al$ be the $n$-tuple $(x_1,\ldots,x_n)$ of coordinates of $\al$ relative to the ordered basis $\mathcal{\be}$. Also, it's easy to verify $T$ is a linear transformation and $T$ is one-one and on-to.
\end{proof}
\begin{remark}
One often identifies isomorphic spaces though the vectors and operations may be quite different.
\end{remark}
\section{Representation of Transformations by Matrices}
Let $V$ be an $n$-dimensional vector space over the field \F and let $W$ be an $m$-dimensional vector space over \F. Let $\mathcal{B} = {\alphavec}$ be an ordered basis for $V$ and $\mathcal{B}'=\{\be_1,\ldots,\be_m\}$ an ordered basis for $W$. If $T$ is any linear transformation from $V$ to $W$, then $T$ is determined by its action on vectors $\al_j$. Each of the $n$ vectors $T\al_j$ is uniquely expressible as a linear combination $$T\al_j=\sum_{i=1}^{m}A_{ij}\be_i$$of the $\be_i$, the scalars $A_{1j},\ldots,A_{mj}$ being the coordinates of $T\al_j$ in the ordered basis $\mathcal{B}'$. Accordingly, the transformation $T$ is determined by the $mn$ scalars $A_{ij}$. The \mbyn matrix $A$ defined by $A(i,j)=A_{ij}$ is called \emph{the matrix of $T$ relative to the pair of ordered bases $\mathcal{B}$ and $\mathcal{B}'$}.\\
If $\al=x_1\al_1+\cdots+x_n\al_n$ is a vector in $V$, then
\begin{align*}
T\al &= T\left(\sum_{j=1}^nx_j\al_j\right) \\&=\sum_{j=1}^nx_jT(\al_j)\\&=\sum_{j=1}^nx_j\sum_{i=1}^mA_{ij}\be_i \\&= \sum_{i=1}^m\left(\sum_{j=1}^nA_{ij}x_j\right)\be_i.
\end{align*}
If $X$ is the coordinate matrix of $\al$ in the ordered basis $\mathcal{B}$, then the computation above shows that $AX$ is the oordinate matrix of the vector $T\al$ in the ordered basis $\mathcal{B}'$, bacause the scalar$$\sum_{j=1}^nA_{ij}x_j$$is the entry in the $i$th row of the column matrix $AX$.\\
We also observe that if $A$ is any \mbyn{} matrix over the field \F, then $$T\left(\sum_{j=1}^nx_j\al_j\right)=\sum_{i=1}^m\left(\sum_{j=1}^nA_{ij}x_j\right)\be_i$$defines a linear transformation $T$ from $V$ into $W$, the matrix of which is $A$, relative to $\mathcal{B},\mathcal{B}'$. We summarize formally:
\begin{theorem}\label{ATT}
Let $V$ be an $n$-dimensional vector spaces over the field \F and $W$ an $m$-dimensional vector space over \F. Let $\mathcal{B}$ be an ordered basis for $V$ and $\mathcal{B}'$ be an ordered basis for $W$. For each linear transformation $T$ from $V$ into $W$, there is an \mbyn{} matrix $A$ with entries in \F{} such that$$[T\al]_{\mathcal{B}'}=A[\al]_{\mathcal{B}}$$for every vector \al{} in $V$. Furthermore, $T\rightarrow A$ is a one-one correspondence between the set of all linear transformations from $V$ into $W$ and the set of all \mbyn{} matrices over the field \F. 
\end{theorem}
\begin{definition}
The matrix $A$ associated with $T$ in Theorem \ref{ATT} is called the \emph{matrix of $T$ relative to the ordered bases $\mathcal{B},\mathcal{B}'$}.
\end{definition}
\begin{remark}
Note that $A$ is the matrix whose columns $A_1,\ldots,A_n$ are given by $$A_j=[T\al_j]_{\mathcal{B}'},\quad j=1,\ldots,n.$$
\end{remark}
\begin{remark}
If $T,U$ are linear transformation from $V$ into $W$ and $A=[A_1,\ldots,A_n],B=[B_1,\ldots,B_n]$ is the matrix of $T,U$ relative to the ordered bases $\mathcal{B},\mathcal{B}'$, then $cA+B$ is the matrix of $cT+U$ relative to $\mathcal{B},\mathcal{B}'$ because
\begin{align*}
cA_j+B_j &= c[T\al_j]_{\mathcal{B}'}+[U\al_j]_{\mathcal{B}'}\\&=[cT\al_j+U\al_j]_{\mathcal{B}'}\\&=[(cT+U)\al_j]_{\mathcal{B}'}.
\end{align*}
\end{remark}
\begin{theorem}
Let $V$ be an $n$-dimensional vector space over the field \F and let $W$ be an $m$-dimensional vector space over \F. For each pair of ordered bases $\mathcal{B},\mathcal{B}'$ for $V,W$ respectively, the function which assigns to a linear transformation $T$ its matrix relative to $\mathcal{B},\mathcal{B}'$ is an isomorphism between the space $L(V,W)$ and the space of all \mbyn{} matrices over the field \F.
\end{theorem}
\begin{proof}
Omit.
\end{proof}
\begin{remark}
If we are considering the representation by matrices of linear transformations of a space into itself, i.e., linear operators on a space $V$. In this case it's convenient to use the same ordered basis in each case, that is, to take $\mathcal{B}=\mathcal{B'}$. We shall then call the representing matrix simply \emph{the matrix of $T$ relative to the ordered basis $\mathcal{B}$}, denoted by $[T]_{\mathcal{B}}$. Note that we have$$[T\al]_{\mathcal{B}}=[T]_{\mathcal{B}}[\al]_{\mathcal{B}}.$$
\end{remark}
Let $V,W,$ and $Z$ be vector spaces over the field \F of respective dimensions $n,m$ and$p$. Let $T$ be a linear transformation from $V$ into $W$ and $U$ a linear transformation from $W$ into $Z$. Suppose we have ordered bases
$$\mathcal{B}=\{\al_1,\ldots,\al_n\},\quad\mathcal{B'}=\{\be_1,\ldots,\be_m\},\quad\mathcal{B''}=\{\gamma_1,\ldots,\gamma_p\}$$for the respective spaces $V,W$ and $Z$.Let $A$ be the matrix of $T$ relative to the pair $\mathcal{B},\mathcal{B'}$ and let $B$ be the matrix of $U$ relative to pair $\mathcal{B},\mathcal{B''}$. It is then easy to see that the matrix $C$ of the transformation $UT$ relative to the pair $\mathcal{B},\mathcal{B''}$ is the product of $B$ and $A$; for , if \al is any vector in $V$ $$[T\al]_{\mathcal{B'}} = A[\al]_{\mathcal{B}}$$$$[U(T\al)]_{\mathcal{B''}}=B[T\al]_{\mathcal{B'}}$$and so $$[(UT)(\al)]_{\mathcal{B''}}=BA[\al]_{\mathcal{B}}$$and hence, by the definition and uniqueness of the representing matrix, we must have $C=BA$. One can also see this by carrying out the computation \begin{align*} (UT)(\al_j)&=U(T\al_j)\\&=U\left(\sum_{k=1}^mA_{kj}\be_k\right)\\&=\sum_{k=1}^mA_{kj}(U\be_k)\\&=\sum_{k=1}^mA_{kj}\sum_{i=1}^pB_{ik}\gamma_i\\&=\sum_{i=1}^p\left(\sum_{k=1}^mB_{ik}A_{kj}\gamma_i\right)\end{align*} so that we must have $$C_{ij} =\sum_{k=1}^mB_{ik}A_{kj}.$$
\begin{theorem}\label{baseth}
Let $V,W,$ and $Z$ be finite-dimensional vector spaces over the field \F; let $T$ be linear transformation from $V$ into $W$ and $U$ a linear transformation from $W$ into $Z$. If $\mathcal{B},\mathcal{B'},$ and $\mathcal{B''}$ are ordered bases for the spaces $V,W,$ and $Z$, respectively, if $A$ is the matrix of $T$ relative to the pair $\mathcal{B},\mathcal{B'}$, and $B$ is the matrix of $U$ relative to the pair $\mathcal{B'},\mathcal{B''}$, then the matrix of the composition $UT$ relative to the pair $\mathcal{B},\mathcal{B''}$ is the product matrix $C=BA$.
\end{theorem}
\begin{remark}
Theorem \ref{baseth} gives a proof that matrix multiplication is assciative. (a proof which requires no calculations)
\end{remark}
\begin{remark}
If $T$ and $U$ are linear operators on a space $V$ and we are representing by a single ordered basis $\mathcal{B}$, then Theorem \ref{baseth} assumes the simple form $[UT]_{\mathcal{B}}=[U]_{\mathcal{B}}[T]_{\mathcal{B}}$. Thus in this case, the correspondence with $\mathcal{B}$ determines between operators and matices is not only a vector space isomorphism but also preserves products. A simple consequence of this is that the linear operator $T$ is invertible if and only if $[T]_{\mathcal{B}}$ is an invertible matrix . For identity operator $I$ is represented by the identity matrix in any  ordered basis, and thus $$UT=TU = I$$ is equivalent to $$[U]_{\mathcal{B}}[T]_{\mathcal{B}}=[T]_{\mathcal{B}}[U]_{\mathcal{B}}=I.$$Of course, when $T$ is invertible $$[T^{-1}]_{\mathcal{B}} = [T]_{\mathcal{B}}^{-1}.$$
\end{remark}
Let $T$ be a linear operator on the finite-dimensional space $V$, and let $$\mathcal{B}=\{\al_1,\ldots,\al_n\}\quad \mathcal{B'}=\{\al_1',\ldots,\al_n'\}$$be two ordered bases for $V$. How are the matrices $[T]_{\mathcal{B}}$ and $[T]_{\mathcal{B'}}$ related?
We observe there is a unique and invertible $n\times n$ matrix $P$ such that $$[\al]_{\mathcal{B}}=P[\al]_{\mathcal{B'}}$$for every vector $\al$ in $V$. It is the matrix $P=[P_1,\ldots,P_n]$ where $P_j = [\al_j']_{\mathcal{B}}$. By definition$$[T\al]_{\mathcal{B}}=[T]_{\mathcal{B}}[\al]_{\mathcal{B}}.$$So we have $$[T\al]_{\mathcal{B}} = P[T\al]_{\mathcal{B'}}.$$ $$[T]_{\mathcal{B}}P[\al]_{\mathcal{B'}}=P[T\al]_{\mathcal{B'}}$$or$$P^{-1}[T]_{\mathcal{B}}P[\al]_{\mathcal{B'}}=[T\al]_{\mathcal{B'}}$$ and so it must be that $$[T]_{\mathcal{B'}}=P^{-1}[T]_{\mathcal{B}}P.$$
\begin{remark}
There is a unique linear operator $U$ which carries $\mathcal{B}$ onto $\mathcal{B'}$, defined by $$U\al_j=\al_j',\quad j=1,\ldots,n.$$This operator $U$ is invertible since it carries a basis for $V$ onto a basis for $V$. The matrix $P$ is precisely the matrix of the operator $U$ in the ordered basis $\mathcal{B}$. For, $P$ is defined by $$\al_j'=\sum_{i=1}^nP_{ij}\al_i$$and since $U\al_j=\al_j'$, this equation can be written $$U\al_j=\sum_{i=1}^nP_{ij}\al_i.$$So $P=[U]_{\mathcal{B}}$, by definition.
\end{remark}
\begin{theorem}
Let $V$ be a finite-dimensional vector space over the field \F, and let $$\mathcal{B} =\{\al_1,\ldots,\al_n\}\quad\mbox{and}\quad\mathcal{B'}=\{\al_1',\ldots,\al_n'\}$$be ordered bases for $V$. Suppose $T$ is a linear operator on $V$. If $P=[P_1,\ldots,P_n]$ is the $n\times n$ matrix with columns $P_j = [\al_j']_{\mathcal{B}}$, then $$[T]_{\mathcal{B'}} = P^{-1}[T]_{\mathcal{B}}P.$$Alternatively, if $U$ is the invertible operator on $V$ defined by $U\al_j=\al_j',j=1,\ldots,n$, then $$[T]_{\mathcal{B'}} = [U]_{\mathcal{B}}^{-1}[T]_{\mathcal{B}}[U]_{\mathcal{B}}$$
\end{theorem}
\begin{definition}[similar]
Let $A$ and $B$ be $n\times n$ matrices over the field \F. We say that $B$ is similar to $A$ over \F if there is an ivertible $n\times n$ matrix $P$ over \F such that $B=P^{-1}AP.$
\end{definition}
\begin{remark}
Similarity is an equivalence relation on the set of $n\times n$ matrices over the field \F.
\end{remark}
\section{Linear Functionals}




















\end{document}